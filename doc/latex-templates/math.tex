% Miscellaneous math packages
\usepackage{amsmath, amssymb, amstext, amsfonts, amscd, amsxtra}

% Math symbol commands
\newcommand{\intd}{\,{\rm d}}   % Symbol 'd' used in integration, such as 'dx'
\newcommand{\diff}{{\rm d}}     % Symbol 'd' used in differentiation
\newcommand{\Diff}{{\rm D}}     % Symbol 'D' used in differentiation
\newcommand{\pdiff}{\partial}   % Partial derivative
\newcommand{\DD}[2]{\frac{\diff}{\diff #2}\left( #1 \right)}
\newcommand{\Dd}[2]{\frac{\diff #1}{\diff #2}}
\newcommand{\PD}[2]{\frac{\pdiff}{\pdiff #2}\left( #1 \right)}
\newcommand{\Pd}[2]{\frac{\pdiff #1}{\pdiff #2}}
\newcommand{\rme}{{\rm e}}      % Exponential e
\newcommand{\rmi}{{\rm i}}      % Imaginary unit i
\newcommand{\rmj}{{\rm j}}      % Imaginary unit j
\newcommand{\vect}[1]{\boldsymbol{#1}}       % Vector typeset in bold and italic
\newcommand{\normvect}{\vect{n}} % Normal vector: n
\newcommand{\dform}[1]{\overset{\rightharpoonup}{\boldsymbol{#1}}}       % Vector for differential form
\newcommand{\cochain}[1]{\overset{\rightharpoonup}{#1}}       % Vector for cochain
\newcommand{\Abs}[1]{\big\lvert#1\big\rvert} % Absolute value (single big vertical bar)
\newcommand{\abs}[1]{\lvert#1\rvert} % Absolute value (single vertical bar)
\newcommand{\Norm}[1]{\big\lVert#1\big\rVert} % Norm (double big vertical bar)
\newcommand{\norm}[1]{\lVert#1\rVert} % Norm (double vertical bar)
\newcommand{\ouset}[3]{\overset{#3}{\underset{#2}{#1}}} % over and under set
% Super/subscript for column index of a matrix, which is used in tensor analysis.
\newcommand{\cscript}[1]{\;\; #1}
\newcommand{\suchthat}{\textit{S.T.\;}} % S.T., such that
% Star symbol used as prefix in front of a paragraph with no indent
\newcommand{\prefstar}{\noindent$\ast$ }      
% Big vertical line restricting the function.
% Example: $u(x)\restrict_{\Omega_0}$
\newcommand{\restrict}{\big\vert}

% Math operators which are typeset in Roman font
\DeclareMathOperator{\sgn}{sgn} % Sign function
\DeclareMathOperator{\erf}{erf} % Error function
\DeclareMathOperator{\Bd}{Bd}   % Boundary of a set or domain, used in topology
\DeclareMathOperator{\Int}{Int} % Interior of a set or domain, used in topology
\DeclareMathOperator{\rank}{rank} % Rank of a matrix
\DeclareMathOperator{\divergence}{div} % Divergence
\DeclareMathOperator{\curl}{curl} % Curl
\DeclareMathOperator{\grad}{grad} % Gradient
\DeclareMathOperator{\tr}{tr} % Trace

% Specify math font
%% Use concmath font
%% \usepackage{concmath}
% For typesetting matrices with dashed separation lines
\usepackage{pmat}
%% Enlarge the default number of allowed columns in matrix environment
\setcounter{MaxMatrixCols}{20}